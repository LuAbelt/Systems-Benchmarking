\documentclass[	runningheads,
				%deutsch, % Tell llncs that the keywords should be in german
				%german,  % Needed for the \ifgerman-command
				a4paper]{llncs}

\usepackage{url}
\usepackage{graphicx}
\usepackage{amssymb}
\usepackage{hyperref}

% Support for special characters like "Umlaute"
\usepackage[utf8]{inputenc}

\usepackage[english]{babel}

\usepackage{glossaries}

\makeglossaries

\loadglsentries{glossary}
%*********************************************************************%
% META                                                                %
%*********************************************************************%
\newcommand{\university}{Saarland University}
\newcommand{\school}{Saarland Informatics Campus}
\newcommand{\thetitle}{Seminar: Systems Benchmarking}
\newcommand{\shorttitle}{Seminar: Systems Benchmarking}
\newcommand{\thedate}{April 15}
\newcommand{\thegermandate}{15. April}

\newcommand{\theforename}{Lukas}
\newcommand{\thesurname}{Abelt}

% Advisors
	\newcommand{\advisor}{Advisors}
	\newcommand{\advisors}{Prof. Sven Apel, \\ Christian Hechtl}

% Title for the seminar
	%\newcommand{\theseminartitle}{"Performance Measurements Before Releases vs. Each Commits"}
	\newcommand{\theseminartitle}{"Using Benchmarking in Productive Development Systems -- Opportunities and Challenges"}

%*********************************************************************%
% THE DOCUMENT                                                        %
%*********************************************************************%

\begin{document}
	%*********************************************************************%
	% TITLE                                                               %
	%*********************************************************************%
	
	% Arabic page numbering
	\mainmatter 
		
	% Title including a subtitle for the name of the seminar
	\title{\theseminartitle \\ \small \thetitle}
	
	% (Optional) In the case that the initial title is too long, the short title will be used
	%\titlerunning{Hauptseminar: Human and Social Factors in Software Engineering}
	
	\author{\theforename\ \thesurname \small \\ \ \\ \advisor : \ \advisors}
	
	% (Optional) This will appear near the page number
	\authorrunning{\shorttitle}
	
	\institute{\school ,\\ \university}
	
	\maketitle
	
	%*********************************************************************%
	% CONTENT                                                             %
	%*********************************************************************%

% General Structure
	% 1. Introduction and Motivation
	% 2. General Definitions
	%		2.1 What is a Benchmark
	%		2.2 What is a continuous Benchmark
	% 3. Experiment
	%		3.1 Experiment Idea&Goals
	%		3.2 Setup
	%		3.3 Evaluation
	%	4. 

	% Introduction
\section{Introduction}
Over the past years system benchmarks have become more prominent in a variety of contexts and use cases. As a result more benchmarking tools became available that are used in both commercial systems and are widely adopted for academic purposes.

In this paper we give an overview of the general concepts of system benchmarks before we consider in detail how classical benchmarking approaches can be used and extended to integrate it into the standard development process of software. Specifically, we take a look at how benchmark can be used to swiftly detect and react to performance changes throughout the development process. In order to do this we will take a look at Change-Point detection algorithms. We will outline and analyse the methodology and opportunities of such approaches while taking a careful look at the challenges that may arise due to this.

To support and evaluate our claims, we will conduct a practical experiment where we will use and compare several banchmarking techniques on a piece of well known, open-source software.
TODO

\section{System Benchmarks}
\label{sec:benchmarking}
The term "Benchmark" is a broad term that can be interpeted and defined in many different ways depending on the context and application domain. We will use this chapter to give a definition of Benchmarking how we will use it throughout this paper.

\subsection{Definition}
\label{ssec:bench_definition}
The term "Benchmark" is used in a variaty of different domains and, according to the Standard Specialization Evalutation Corporation, historically stems from a physical marking that was used as a refernce point on a workbench to check the length of the produced pieces\footnote{SPEC Glossary: \url{https://www.spec.org/spec/glossary/\#benchmark}}. However throughout this paper we are only interested in system benchmarks within the context of software systems and will therefore also refer to the definition of a benchmark as given by Kounev et al.: "a tool coupled with a methodology for the evaluation and comparison of systems or components with respect to specific characteristics, such as performance, reliability or security." (\cite{Kounev} p. 4). As already given from this definition benchmarks can serve as an evaluation for different characteristics. We will more closely look at these in \autoref{ssec:bench_classification}. Throughout this paper we will also mostly refer to the "system or component" that is being benachmarked as the \gls{sut}.

\subsection{Classification of Benchmarks}
\label{ssec:bench_classification}
Benchmarks can be classified in a variety of different means. In this section we will give a small overview of the different types of benchmarks there are with respect to the following questions:
\begin{enumerate}
	\item What overall goal does a benchmark serve?
	\item What qualities are evaluated by an benchmark?
	\item How can a benchmark be performed (Benchmarking strategies)
\end{enumerate}

In the broadest sense, benchmarks can be divided into competetive and non-competetive benchmarks. The main motivation of competetive benchmarks is hereby the development of standardized quality criteria that can be used to compare different systems. Certain sources also emphasize this importance on the competetiveness by providing a more detailled definition of the term benchmark, i.e. "a standard tool for competetive evaluation and comparison of competetive systems" (\cite{kistowski2015} p. X).

For non-competetive benchmarks Kounev et al. Further distiguish between the two categories of "rating tools" and "research benchmark. 

\subsection{Evaluating Benchmarks}

\section{Detecting Performance Changes}

\section{Continuous Benchmarking}
\subsection{Definition}
\subsection{Usages}
\subsection{Challenges}

\section{Experiment}
	\subsection{Experiment Idea \& Goals}
	\subsection{Experiment Setup}
	\subsection{Experiment Results}

\section{Evaluation of Performance Results}

\section{Reflection}

\section{Future Work}
	
	%*********************************************************************%
	% APPENDIX                                                            %
	%*********************************************************************%
	
	% Insert the appendix here. You can alternatively include files via: \include{pathToFile}
	
	%*********************************************************************%
	% LITERATURE                                                          %
	%*********************************************************************%
	% As a recommendation JabRef might be a usefull tool for this section. Use myRefs.bib therefore
	\phantomsection
	\bibliographystyle{splncs03}
	\bibliography{literature}	
\end{document}
